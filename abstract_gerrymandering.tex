\documentclass[twoside,11pt,reqno]{amsart}
\usepackage{amsmath,amssymb,amscd,mathrsfs,epic,wasysym,latexsym,tikz,mathrsfs,cite,hyperref}
\makeatletter

\hfuzz 3pt
\vfuzz 2pt

\textheight 200mm
\textwidth 135mm

\raggedbottom

\begin{document}

\begin{center}
{\scshape{A Merry Gander at Gerrymandering}}
\end{center}
\vspace{.5cm}
In a startling ruling this month, federal judges voided three Texas districts on the grounds 
that the distribution unfairly diluted minority votes to diminish their representation. 
This result highlights the need for a metric that quantifies the ‘fairness’ of a proposed 
district. 
With congressional and state redistricting approaching, this has become a pressing concern nationwide. 

While arguments can be made for and against given districts, we need a more concrete 
and well-defined method for evaluating these characteristics in a legal framework. 

Our contribution is to apply a Metropolis-Hastings algorithm with probabilities taken from a 
function space over metrics including compactness (geographic shape), compliance (Voting Rights Act), and efficiency (party distribution).
For each function we find an optimal redistricting, and compare with results from other functions in the space. 
This allows us to identify characteristics that lead to a more fair district and quantify differences between redistricting plans.
\end{document}
